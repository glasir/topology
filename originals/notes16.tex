\documentclass[10pt,reqno]{amsart}
\usepackage{amsmath}
\usepackage{amsthm}
\usepackage{amssymb}
\usepackage{amsfonts}
\usepackage{latexsym}
\usepackage{verbatim}
%\usepackage{graphicx}

%%  Renewed
\renewcommand{\phi}{\varphi}
\renewcommand{\Re}[1]{\operatorname{Re} #1 }
\renewcommand{\Im}[1]{\operatorname{Im} #1}

%% New Commands
\newcommand{\C}{\mathbb{C}}
\newcommand{\h}{\mathcal{H}}
\renewcommand{\S}{\mathcal{S}}
\newcommand{\Z}{\mathbb{Z}}
\newcommand{\N}{\mathbb{N}}
\newcommand{\R}{\mathbb{R}}
\newcommand{\Q}{\mathbb{Q}}
\newcommand{\D}{\mathbb{D}}
\newcommand{\F}{\mathbb{F}}
\newcommand{\cl}{\operatorname{cl}}
\newcommand{\ran}{\operatorname{ran}}
\newcommand{\vecspan}{\operatorname{span}}
\newcommand{\norm}[1]{\| #1 \|}
\newcommand{\inner}[1]{\langle #1 \rangle}

%%  Matrices
\newcommand{\minimatrix}[4]{\begin{pmatrix} #1 & #2 \\ #3 & #4 \end{pmatrix}  }
\newcommand{\megamatrix}[9]{\begin{pmatrix} #1 & #2 & #3 \\ #4 & #5 & #6 \\ #7 & #8 & #9\end{pmatrix}  }

%\renewcommand{\labelenumi}{(\roman{enumi})}

\newcommand{\twovector}[2]{\begin{pmatrix} #1\\#2 \end{pmatrix} }
%\newcommand{\threevector}[3]{\begin{pmatrix} #1\\#2\\#3 \end{pmatrix} }

%%%
%%% Theorem Styles
%%%
\newtheorem{Proposition}{Proposition}
\newtheorem{Corollary}{Corollary}
\newtheorem{Theorem}{Theorem}
\newtheorem*{Small Fact}{Small Fact}
\newtheorem*{Small Fact about Basis}{Small Fact about Basis}
\newtheorem*{Tiny Fact about Projection Maps}{Tiny Fact about Projection Maps}
\newtheorem*{Thm}{Theorem}
\newtheorem{Lemma}{Lemma}
\theoremstyle{definition}
\newtheorem*{Definition}{Definition}
\newtheorem{Example}{Example}
\newtheorem*{Remark}{Remark}
\newtheorem*{Question}{Question}
\newtheorem{Problem}{Problem}
\newtheorem{Analogous Theorem}{Analogous Theorem}


\allowdisplaybreaks
\begin{document}
\begin{center}
\huge Topology
March 3, 2010
\normalsize
Madeline Wyse

\vspace{.2in}
$\mathbf{EXAM:}$
4 Questions. 2 Hours. Starts at 8:00 AM.
Covers up through Quotient Spaces (no Product Spaces).
We should actually describe/name the theorems/homework results we cite.
\end{center}
\bigskip
$\mathbf{PART \ III: \ DISTINGUISHING  \ SPACES}$
\begin{Definition}
A $\mathbf{topological \ property}$ (top. prop.) is a property of a topological spaces that is preserved by homeomorphisms. 
\end{Definition}
\begin{center} $\mathbf{List \ of \ Topological \ Properties \ thus \ Far}$ \end{center}
\begin{enumerate}
\item Cardinality of $X$ 
\item Cardinality of $F_x$
\item Metrizability (we proved this in the homework)
\item Discreteness 
\item Indiscretness
\end{enumerate}
\vspace{.1 in}
$\mathbf{Rachel's \ Lemma}$
Let $(X, F_X) $ and $(Y, F_Y)$ be topological spaces and $f: X \rightarrow Y$ be  an open bijection. If $F_X$ is the discrete topology, then $F_Y$ is the discrete topology.
\begin{proof}
Let $y \in Y$. Since $f$ is surjective, there exists $x \in X$ such that $f(x) = y$. Since $\{x\} \in F_X$ and $f$ open, $f(\{x\}) \in F_Y$. Thus all singletons in are elements of $F_Y$ and all sets in $Y$ are unions of singletons and hence elements of $F_Y$, so every set is an open set and $F_Y$ is the discrete topology. 
\end{proof}
$\mathbf{Daniel's \ Lemma}$. 
Let $(X, F_X) $ and $(Y, F_Y)$ be topological spaces and $f: X \rightarrow Y$ be  a continuous bijection. If $F_X$ is the indiscrete topology, then $F_Y$ is the indiscrete topology.
\begin{proof}
Let $U \in F_Y$. WTS $U = Y$ or $U = \emptyset$. Since $f$ is continuous, $f^{-1}(U) \in F_X$. Therefore $f^{-1}(U) = X$ or $\emptyset$. \\
Suppose $f^{-1}(U) = X$. Then $U = f(f^{-1}(U)) = f(X) = Y$ since $f$ surjective.\\
Suppose $f^{-1}(U) = \emptyset$. Then $U = \emptyset$. Therefore $F_Y$ is the indiscrete topology. 
\end{proof}
$\mathbf{Corollaries}$
If $(X, F_X) $ and $(Y, F_Y)$ topological spaces and $f: X \rightarrow Y$ a homeomorphism, then 
\begin{enumerate}
\item $F_X$ the discrete topology $\Longrightarrow$ $F_Y$ the discrete topology
\item $F_X$ the indiscrete topology $\Longrightarrow$ $F_Y$ the indiscrete topology
\end{enumerate}
\vspace{.1in}
Unfortunately, even with these five lovely Topological Properties, we can't yet distinguish a circle from a line. Clearly there is more work to do. \\
\newpage
$\mathbf{Non \ Example}$
Distance is not a topological property. A big circle $is$ homeomorphic to a little circle.\\
\vspace{1in}
\begin{center} $\mathbf{Chapter \ 7: \  Completeness}$ \end{center}
\begin{Definition}
Let $(X, F_X)$ be a topological space and $S \subseteq X$.\\ 
We say $\{U_j : j \in J\}$ is an $\mathbf{open \ cover}$ of $S$ if for all $j \in J$, $U_j \in F_X$ and $S \subseteq \bigcup_{j \in J}U_j$.\\ 
We say $\{U_j : j \in K\}$ is a $\mathbf{subcover}$ if $K \subseteq J$ and $S \subseteq \bigcup_{j \in K} U_j$.\\
We say $S$ is $\mathbf{compact}$ if every open cover of $S$ has a finite subcover. 
\end{Definition}
\vspace{.1in}
$\mathbf{IMPORTANT \ WARNING}$\\
We learned in 131 that in $\R^{n}$, a set if compact $\Longleftrightarrow$ it is closed and bounded. THIS IS NOT TRUE IN TOPOLOGICAL SPACES. DO NOT TRY TO USE IT.
\vspace{.1in}
\bigskip

$\mathbf{ Example}$
Is the set $(0,1)$ compact in 
\begin{enumerate}
\item $\R$ with the discrete topology?
\item $\R$ with the half-open topology?
\item $\R$ with the finite complement topology?
\end{enumerate}
Answers:\\
(1) No. Take the open covering $\{B_{1/2}(x) : x \in (0,1)\}$. If we remove any of these balls, the corresponding $x$ will no longer be covered. However, there are clearly uncountably infinitely many balls. \\\\
(2) No. Take the open covering $\{[\frac{1}{n}, 1) : n \in \N\}$ so that $(0,1) = \bigcup_{n \in \N}([\frac{1}{n}, 1)$. There is not finite subcover. \\\\
(3) Yes. Suppose we have a cover of $(0,1)$. Take any element of the cover, say $U_{47}$. $U_{47}$ is missing at most finitely may elements of $(0,1)$ since its complement is finite. For each element $x \in \R \setminus U_{47}$, select one $U_{j_x}$ containing $x$. The set $\{U_{47}\} \cup \{U_{j_x}: x \in \R \setminus U_{47}\}$ is a finite open subcover. 
\begin{Analogous Theorem}
Let $(X, F_X)$ and $(Y, F_Y)$ be topological spaces, $S \subseteq X$, and $f : X \rightarrow Y$ continuous. If $S$ is compact then $f(S)$ is compact. 
\end{Analogous Theorem}
\begin{proof}
\begin{enumerate}
\item Take a covering of $f(S)$. 
\item Applying $f^{-1}$ to these sets, we pull back to an open covering of $S$.
\item Take a finite subcover. 
\item Push these forward again. We have a finite cover of $f(S)$. 
\end{enumerate}
\end{proof}
As a corollary we have $\mathbf{Topological \ Property \ 6:}$ Compactness
\begin{Analogous Theorem}
Any closed subset of a compact space is compact. 
\end{Analogous Theorem}
\begin{proof}
\begin{enumerate}
\item Take an open cover of $S$
\item Add $X \setminus S$. Now we have an open cover of $X$. 
\item Take a finite subcover.
\item Remove $X \setminus S$. Not we have a finite subcover of $S$. 
\end{enumerate}
\end{proof}
\begin{Analogous Theorem}
Let $(X, F_X)$ be compact and $f : X \rightarrow \R$ be continuous. Then $f$ has a max value and a min value. 
\end{Analogous Theorem}
\begin{proof}
\begin{enumerate}
\item $f(X)$ is compact by Analogous Theorem 1
\item Therefore $f(X)$ is closed and bounded
\item Therefore $f(X)$ has a lub and a glb (ie a max and a min)
\end{enumerate}
\end{proof}
$\mathbf{Small \ Fact \ about \ Compactness}$
Let $(X, F_X)$ be a topological space. Suppose every open cover of $S \subseteq X$ made up of basis elements has a finite subcover. Then $X$ is compact. 
\begin{proof}
Let $\{U_j : j \in J\}$ be an open cover of $S$ and $\beta$ be a basis of $S$. For every $j \in J$, $U_j = \bigcup_{i \in I_j}B_i$ where for every $i \in I_j$ $B_i \in \beta$. Hence $\{B_i : i \in I_j \text{ and } j \in J\}$ is an open cover of $S$. By hypothesis, we have a finite subcover, $\{B_i : i \in K_j \text{ and } j \in K\}$ where $K \subseteq J$ and $K$ is finite, and for all $j \in K$, $K_j \subseteq I_j$ and $K_j$ is finite. Now note that given any $j \in K$, for every $i \in K_j$ $B_i \subseteq U_j$.  Hence $$X = \bigcup_{j\in K} \left[ \bigcup_{i \in K_j} B_i \right] \subseteq \bigcup_{j\in K} U_j \subseteq X$$ and $\{U_j : j\in K\}$ is a finite subcover. 
\end{proof}
\end{document}