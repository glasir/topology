\documentclass{article}

\usepackage{amssymb}
\newtheorem{Thm}{Theorem}
\newtheorem{Ex}{Example}
\newtheorem{Def}{Definition}
\newtheorem{smallf}{Small Fact}

\begin{document}

\begin{Ex}

Let $M$ be a set and $d$ a function which satisfies property $2$ and not $1$ of the definition of a metric space.

$$d(a,b) = 0 \quad \forall a,b \in M$$ and $M$ has at least 2 points.

\end{Ex}

Goal: talk about continuity in metric spaces (rather than just the reals)

\begin{Def}

Let $(M_1, d_1)$ and $(M_2, d_2)$ be metric spaces and $a \in M_1$. Let $f:M_1 \rightarrow M_2$. We say $f$ is continuous at a if $\forall \epsilon > 0$ there exists a
$\delta > 0$ such that $\forall x\in M_1$ with $d_1 (x,a) <\delta$ then $d_2 (f(x), f(a))<\epsilon$.

\end{Def}

\bigskip

\bigskip

\bigskip

\bigskip


\begin{Def}

Let (M,d) be a metric space and $a\in M$. The open ball of radius $\epsilon >0$ is $B_\epsilon (a) = \{x\in M | d(x,a)<\epsilon \}$

\end{Def}

So in terms of open balls:

$f$ is continuous at a if $\forall\epsilon >0$ there exists a $\delta >0$ such that if $x\in B_\delta (a)$ then $f(x) \in B_\epsilon (f(a))$

In other words $f(B_\delta (a)) \subseteq B_\epsilon (f(a))$


Open balls don't always look like open balls...

\begin{Ex}

$M =$ upper half of plane with usual distance

(picture) An open ball can look like a semicircle if it rests on the x-axis.

\end{Ex}

\begin{Ex}

$\mathbb{R}$ with the discrete metric.

$$B_1 (47) = {47}$$

$$B_2 (47) = \mathbb{R}$$

\end{Ex}


\begin{Ex}

Comb metric

$B_{\frac{1}{2}} ((0,1))$ is just an interval along the y-axis starting at $(0,1)$

$B_2 ((0,1)$ This is everything on the comb below the line of slope $-1$ that connects points $(0,1)$ and $(1,0)$. Since it's an open ball, the line is not contained in
the ball.

\end{Ex}

Note: Balls are not closed under $\bigcup$ and $\bigcap$

\begin{Def}

Let $(M,d)$ be a metric space. A set $U \subseteq M$ is said to be open if $\forall a\in U$ there exists an $\epsilon >0$ such that $B_\epsilon (a) \subseteq U$

\end{Def}


Question: What are all the open sets in the discrete metric?
Answer: All subsets of M, whatever M is!

\begin{smallf} (a.k.a. ``not hard to prove but not teeny-weeny and seemingly obvious")

In a metric space, open balls are open sets.

\end{smallf}

$proof:$
Let $a \in M$ and $\epsilon >0$.

WTS: $B_\epsilon (a)$ is open
WTS: $\forall x \in B_\epsilon (a)$ there exists an $r>0$ such that $B_r (x) \subseteq B_\epsilon (a)$

Let $x \in B_\epsilon (a)$

Let $r < \epsilon - d(x,a)$ and $r > 0$

Claim: $B_r (x) \subseteq B_\epsilon (a)$

Let $y \in B_r (x)$

$d(y,a)\leq d(y,x) + d(x,a)$

$< \epsilon - d(x,a) + d(x,a)$

$= \epsilon$

Therefore $B_r (x) \subseteq B_\epsilon (a)$



\begin{Thm} (``open sets are nice" theorem)

Let $F$ be the family of open sets in a metric space $(M,d)$. Then:

1) $M$, $\emptyset \in$ $F$

2) If U, V $\in$ F then U $\cap$ V $\in$ F

3) If $\forall i \in I$, $U_i \in F$ then $\bigcup_{i \in I} U_i \in F$


\end{Thm}

Define: $\bigcup_{i \in I} U_i = \{x \in U_i | i \in I\}$

$proof:$
1) Let $x\in M$, and $\epsilon = 47$

$B_{47} (x) = \{y\in M|d(x,y)<47\} \subseteq M$

and $\forall x\in \emptyset$ there exists an $\epsilon >0$ such that $B_\epsilon (x) \subseteq \emptyset$

\medskip

2) Let $x\in U\cup V$

Let $r_1 >0$ such that $B_{r_1} (x) \subseteq U$

Let $r_2 >0$ such that $B_{r_2} (x) \subseteq V$

Let $r = \min\{r_1, r_2 \}$

$B_r (x) \subseteq B_{r_1} (x) \subseteq U$

$B_r (x) \subseteq B_{r_2} (x) \subseteq V$

Therefore $B_r (x) \subseteq U \cap V$

\medskip

3) Let $x \in \cup_{i\in I} U_i$

WTS: There exists an $r>0$ such that $B_r (x) \subseteq \cup_{i\in I} U_i$

$x\in \cup_{i\in I} U_i$ implies that there exists an $i_o \in I$ such that $x\in U_{i_o}$ so there exists an $r>0$ such that $B_r (x) \subseteq U_{i_o}$

$B_r (x) \subseteq U_{i_o} \subseteq \bigcup_{i\in I} U_i$


\medskip

\begin{Def}

If f:X $\rightarrow$ Y and $U \subseteq Y$ then $f^{-1} (U) = \{x\in X |f(x) \in U\}$

\end{Def}

\begin{Thm}

Let $(M_1, d_1)$ and $(M_2, d_2)$ be metric spaces and $f: M_1 \rightarrow M_2$. Then $f$ is continuous if and only if $\forall U \subseteq M_2$ which is open in $(M_2,
d_2)$, $f^{-1} (U)$ is open in $(M_1, d_1)$.

\end{Thm}

$proof:$
$(\Rightarrow)$ Suppose $f$ is continuous. Let $U \subseteq M_2$ be open.

WTS: $\forall x \in f^{-1} (U)$ there exists an $r>0$ such that $B_r (x) \subseteq f^{-1} (U)$

Let $x\in f^{-1} (U)$

$f(x) \in U$ so since U is open, there exists an $\epsilon >0$ such that $B_\epsilon (f(x)) \subseteq U$

Since $f$ is continuous there exists an $r>0$ such that \[f(B_r (x)) \subseteq B_\epsilon (f(x)) \subseteq U\]

So $f^{-1} (f(B_r (x))) \subseteq f^{-1} (U)$

Therefore $B_r (x) \subseteq f^{-1} (f(B_r (x))) \subseteq f^{-1} (U)$

\smallskip

$(\Leftarrow)$ Suppose that $\forall U$ open in $(M_2, d_2) f^{-1} (U)$ is open in $(M_1, d_1)$

WTS: $f$ is continuous at every point.

Let $a\in M_1$

WTS: $\forall \epsilon >0$ there exists a $\delta > 0$ such that $f(B_\delta (a)) \subseteq B_\epsilon (f(a))$



\end{document}
