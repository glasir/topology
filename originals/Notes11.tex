\documentclass[11pt]{amsart}
\usepackage{geometry}                % See geometry.pdf to learn the layout options. There are lots.
\geometry{letterpaper}                   % ... or a4paper or a5paper or ... 
%\geometry{landscape}                % Activate for for rotated page geometry
\usepackage[parfill]{parskip}    % Activate to begin paragraphs with an empty line rather than an indent
\usepackage{graphicx}
\usepackage{amssymb}
\usepackage{epstopdf}
\usepackage[all,2cell]{xy}
\DeclareGraphicsRule{.tif}{png}{.png}{`convert #1 `dirname #1`/`basename #1 .tif`.png}

\newtheorem*{theorem}{Theorem}
\newtheorem*{defn}{Definition}
\newcommand{\ex}{ {\bf Example.} }

\newcommand{\onto}{\twoheadrightarrow}
\newcommand{\R}{\mathbb R}

\begin{document}
\begin{center}
\LARGE{Math 147 - Topology}

\large{Erica Flapan}

\mbox{ }

\large{Notes - 19 February 2010}
\end{center}

\mbox{ }

{\bf Important Lemma about Quotients: } 
Let $(X, F_X)$ and $(Z, F_Z)$ be topological spaces and let $Y$ be a set.  Suppose $f:X\onto Y$.  Let $g:(Y, F_f)\to(Z, F_Z)$.  Then $g$ is continuous if and only if $g\circ f$ is continuous.

{\bf Proof: }

$(\Rightarrow)$: Suppose that $g$ is continuous.  Since $f$ is a quotient map, it is continuous.  Therefore, $g\circ f$ is a composition of continuous functions, and so $g\circ f$ is continuous.

$(\Leftarrow)$:  Suppose, now, that $g\circ f$ is continuous.  Let $U\in F_Z$.  Then $(g\circ f)^{-1}(U)\in F_X$.  Therefore, $(g\circ f)^{-1}(U) = f^{-1}(g^{-1}(U)) \in F_X$.
 
 Recall that $F_f = \{ V\subseteq Y : f^{-1}(V)\in F_X\}$.  So, $g^{-1}(U) \in F_f$, and $g$ is continuous.  \begin{flushright}$\blacksquare$\end{flushright}

\mbox{ }

Now, a non-example.  Define functions $f:\R\setminus\{1\}\to\R$ and $g:\R\to\R$ by
\[ f(x) = \begin{cases} x^2 & x\in(-\infty, 1) \\ \frac{-1}{x-1} & x\in (1, \infty) \end{cases} \]
\[ g(x) = \begin{cases}  1 &  x\ge 0 \\  -1 & x < 0 \end{cases} \]
Then $g\circ f : \R\setminus\{1\}\to\R$ is defined by
\[ (g\circ f)(x) = \begin{cases} -1 & x > 1 \\ 1 & x < 1 \end{cases} \]
which doesn't seem to be continuous.  But according to the lemma, $g\circ f$ should be.  

The solution is, of course, that $g$ actually \emph{is} continuous - the topology of its domain isn't the usual topology.  Instead, it has the topology induced by $f$.  That is, 
\[F_f = \{ U\subseteq \R : f^{-1}(U)\text{ open in }\R\setminus\{1\} \}\]
More precisely, we have $g^{-1}(\{ 1\}) = [0, \infty)$ and $f^{-1}([0, \infty)) = (-\infty,1)$, which is open.

Also, $g^{-1}( \{-1 \} ) = (-\infty,0)$ and $f^{-1}((-\infty, 0)) = (1, \infty)$, which again is open.

Therefore, $g$ is continuous (as we only need to look at the range of $g$, which is $\{-1, 1\}$).

\mbox{ }

It would be nice to be able to say precisely whether equivalence relations are preserved by homeomorphisms.  
\begin{theorem}  Let $(A, F_A)$ and $(B, F_B)$ be topological spaces and $f:A\to B$ be a homeomorphism.  Let $\sim_A$ and $\sim_B$ be equivalence relations on $A$ and $B$, respectively, such that $x\sim_A x'$ if and only if $f(x) \sim_B f(x')$.

Then $A/\sim_A \equiv B/\sim_B$.
\end{theorem}
{\bf Proof: }

We want to show that there is some function $g$ such that the following diagram commutes:
\[ \xymatrix{ A \ar[rr]^f \ar[dd]_{\pi_A} && B \ar[dd]^{\pi_B} \\ &&\\C \ar @{.>}[rr]_g & & D } \]
Define $g:A/\sim_A \to B/\sim_B$ by $g\left([x]_A\right) = \left[ f(x) \right]_B$.  To show it's a homeomorphism:
\begin{itemize}
\item {\bf Well-defined:} Suppose $[y]_A = [z]_A$.  Then 
\[y\sim_A z\qquad\Rightarrow\qquad f(y) \sim_B f(z) \qquad\Rightarrow \qquad[f(y)]_B = [f(z)]_B]\]
and so $g$ is well-defined.
\item {\bf 1-to-1:} Suppose that $[x]_A$ and $[y]_A$ are such that $g([x]_A) = g([y]_A)$.  Then
\[ [f(x)]_B = [f(y)]_B\quad\Rightarrow\quad f(x)\sim_B f(y)\quad\Rightarrow\quad x\sim_A y \quad\Rightarrow\quad [x]_A = [y]_A\]
and so $g$ is 1-to-1.
\item {\bf Onto: } Let $[y]_B\in B/\sim_B$, and let $x = f^{-1}(y)$.  Then
\[ g([x]_A) = [f(x)]_B = [y]_B \]
and $g$ is onto.
\item {\bf Continuous: } We want to show that $g\circ \pi_A$ is continuous, so we can use the Important Lemma.  By the definition of $g$, $g\circ \pi_A = \pi_B\circ f$.  Since $f$ is a homeomorphism, it is continuous; $\pi_B$ is a quotient map and is thus continuous.  So $\pi_B\circ f$ is continuous.  Since $g\circ \pi_A = \pi_B\circ f$, we know that $g\circ \pi_A$ is continuous.  As $\pi_A$ is a quotient map, it is continuous, so from the Important Lemma, $g$ is continuous.
\item {\bf Open: }  We can show equivalently that $g^{-1}$ is continuous instead, since $g$ is a bijection.
\begin{align*}
g\circ\pi_A &= \pi_B\circ f \\
g^{-1}\circ(g\circ\pi_A)\circ f^{-1} &= g^{-1}\circ(\pi_B\circ f)\circ f^{-1} \\
\pi_A\circ f^{-1} & =g^{-1}\circ \pi_B
\end{align*}
From here, the argument is eerily similar to the argument for continuity above.

Note that $\pi_A$ is a quotient map and $f^{-1}$ s a homeomorphism, so both are continuous.  As such, $\pi_A\circ f^{-1}$ is continuous, and therefore so is $g^{-1}\circ \pi_B$.  Since $\pi_B$ is a quotient map, it is continuous; from the Important Lemma, $g^{-1}$ is continuous.  Therefore, $g$ is open.
\end{itemize}
Therefore, $g$ is a homeomorphism. Hurrah!\begin{flushright}$\blacksquare$\end{flushright}
\end{document}  