\documentclass[11pt,letterpaper,boxed]{article}
\usepackage[usenames]{color}
\usepackage{amsthm,amssymb,amsmath}
\newtheorem*{example}{Example}
\newtheorem*{defn}{Definition}
\newtheorem*{theorem}{Theorem}
\newtheorem*{smallfact}{Small Fact}
\newtheorem*{lemma}{Lemma}
\newcommand{\R}{\mathbb{R}}
\newcommand{\Q}{\mathbb{Q}}
\begin{document}
Math147: Topology Notes: 4/14/10\\

\begin{center}{\Large }
\end{center}
\normalsize
\vspace{.2in}

Recall from the previous lecture that we were attempting to prove the following theorem:
\begin{theorem}
Let $x,y\in X$.  If there is a path $f$ in $X$ from $x$ to $y$, then $\pi_1(X,x)\cong \pi_1(X,y)$.
\end{theorem}

And we defined a helpful map:
\begin{defn} Let $u_f:\pi_1(X,x)\to \pi_1(X,y)$ by $u_f([g])=[\overline{f}*g*f]$.
\end{defn}

We wish to show that $u_f$ is, in fact, an isomorphism between $\pi_1(X,x)$ and $\pi_1(X,y).$  As we showed last lecture (see previous notes) that $u_f$ is well defined and one-to-one, it suffices to show that $u_f$ is onto and a homomorphism.
\vspace{.2in}

\begin{proof}

\vspace{.1in}

{\bf onto:}
Let $[g]\in \pi_1(X,y)$.  Then $[f*g*\overline{f}]\in \pi_1(X,x)$, and $u_f([f*g*\overline{f}])=u_f([\overline{f}*(f*g*\overline{f})*f])$, which by associativity, inverses, and identity, is precisely $[g]$.  Therefore $u_f$ is onto.
\vspace{.2in}

{\bf homomorphism:}
Let $g,h\in \pi_1(X,x)$.  Then \[\begin{array}{cccc} u_f([g])u_f([h])& =& [\overline{f}*g*f][\overline{f}*h*f] & \\ & = & [\overline{f}*g*h*f] =& u_f([g*h]).\end{array}.\]

It follows that $u_f$ is in fact an isomorphism.
\end{proof}


Our overall goal is to show that the fundamental group is a topological invariant.  In general, a 
\emph{topological invariant} is a ``thing'' that takes a value on topological spaces such that every homeomorphic space takes the same value.  These invariants are used to distinguish topological spaces.

So we want to show now that the Fundamental Group is a topological property.

\begin{defn}  Let $\phi:X\to Y$ be continuous, and $\phi(x_0)=y_0$.  Define $\phi_*:
\pi_1(X,x_0)\to \pi_1(X,y_0)$ by $\phi_*([f]_X)=[\phi(f)]_Y.$

We say that $\phi_*$ is \underline{induced} by $\phi$.
\end{defn}

\vspace{.3in}
\begin{smallfact}[About Induced Maps]  Let $\phi:X\to Y$ and $\phi(x_0)=y_0$.  Then $\phi_*$ is well defined.
\end{smallfact}

\begin{proof}
Let $f$ and $g$ be loops in $X$ based at $x_0$ such that $f\sim g$.  Then there exists $F:I\times I\to X$, a path homotopy from $f$ to $g$.  \vspace{.4in}

Consider $\phi(F): I\times I \to Y$.  We see that $\phi$ is a composition of continuous functions, so is itself continuous.  
\vspace{.1in}

Also, observe that  \[\begin{array}{c} \phi(F(s,0))=\phi\circ f(s)\\ \phi(F(s,1))=\phi\circ g(s)\\ \hline \phi(F(0,t))=\phi(x_0)=y_0\\ \phi\circ F(1,t)=\phi(x_0)=y_0 \end{array}\]

Thus $\phi(F)$ is a homotopy between $\phi\circ g$ and $\phi\circ f$.  
So $\phi\circ g\sim \phi\circ f$-- and $\phi_*$ is well defined.

\end{proof}

\begin{lemma} Let $\phi:X\to Y$ be continuous and $\phi(x_0)=y_0$.  Then $\phi_*$ is a homomorphism.
\end{lemma}
\begin{proof} Let $[f],[g]\in \pi_1(X,x_0).$  We want to show that $\phi_*([f]_X[g]_X)=\phi_*([f]_X)\phi_*([g]_X$.

Observe that \[\begin{array}{c}
\phi_x([f]_X[g]_X)=\phi_*([f*g]_X)=[\phi(f*g)]Y\\
\phi\circ (f*g)(x)=\phi_*\begin{cases} f(2s), &s\in [0,1/2]\\ g(2s-1) & s\in [1/2,1]
\end{cases}\\
= \begin{cases} \phi\circ f(2s) & s\in [0,1/2]\\ \phi\circ g(2s-1) & s\in [1/2,1]\end{cases}=(\phi\circ f)*(\phi\circ g)(s).
\end{array}\]

So $\phi\circ f * g=\phi\circ f * \phi\circ g$ and in particular, \[ \begin{array}{cc}  & \phi(f*g)]_Y  =[(\phi\circ f)*(\phi\circ g)]_Y\\ & =[\phi\circ f]_Y[\phi\circ g]_Y=\phi_*([f]_X)\phi_*([g]_X)\end{array}.\]  This concludes the proof of the lemma.

\end{proof}

\vspace{.3in}
\begin{theorem}
Let $\phi:X\to Y$ be a homeomorphism and $\phi(x_0)=y_0$.  Then $\phi_*$ is an isomorphism.
\end{theorem}

With the previous lemma we showed that $\phi_*$ is a homomorphism.  It remains to show that $\phi_*$ is a bijection.

\begin{proof}

{\bf one-to-one}  Let $[f]_X, [g]_X\in \pi_1(X,x_0)$ such that $\phi_* ([f]_X)=\phi_*([g]_X)$.  Then by definition of $\phi_*$, $[\phi\circ f]_Y=[\phi\circ g]_Y$.  

Observe that $\phi\circ f\sim_Y \phi\circ g$, so there exists a path homotopy $F$ from $\phi\circ f$ to $\phi\circ g$.  Also note that, as $\phi$ is a homeomorphism, $\phi^{-1}:Y\to X$ is continuous.

Hence $\phi^{-1}\circ F:I\times I \to Y$ is a path homotopy from $\phi^{-1}\circ \phi\circ f$ to $\phi^{-1}\phi\circ g.$ Since $g$ is bijective, this is a path homotopy from $f$ to $g$.

\vspace{.2in}
{\bf onto:}
Recall that $\phi_*:\pi_1(X,x_0)\to \pi_1(Y,y_0)$.  Let $[f]\in \pi_1(Y,y_0)$.  Then $[\phi^{-1}(f)]_X\in \pi_1(X,x_0)$, because $\phi^{-1}$ is continuous.

$\phi_*([\phi^{-1}(f)]_X)=[\phi\circ \phi^{-1}(f)]_Y$, and as $\phi$ is a bijection, this is $[f]_Y$.  

This illustrates that $\phi_*$ is bijective, and hence is an isomorphism between $X$ and $Y$.  
\end{proof}

\vspace{.3in}
\begin{smallfact}[About Induced Homos]
\begin{itemize}
\item If $\phi:X\to Y$, $\psi:Y\to Z$ are continuous, then $(\psi\circ \phi)_*=\psi_*\circ \phi_*$.

\item If $i:X\to X$ is the identity, then $i_*$ is the identity isomorphism.

\item Let $\phi:X\to Y$ be a continuous map, and $f$ a path in $X$ from $p$ to $q$.  Then $\phi_*\circ u_f=u_{\phi(f)}\circ \phi_*$.
\end{itemize}
\end{smallfact}

\end{document}